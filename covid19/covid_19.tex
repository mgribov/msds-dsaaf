% Options for packages loaded elsewhere
\PassOptionsToPackage{unicode}{hyperref}
\PassOptionsToPackage{hyphens}{url}
%
\documentclass[
]{article}
\usepackage{amsmath,amssymb}
\usepackage{iftex}
\ifPDFTeX
  \usepackage[T1]{fontenc}
  \usepackage[utf8]{inputenc}
  \usepackage{textcomp} % provide euro and other symbols
\else % if luatex or xetex
  \usepackage{unicode-math} % this also loads fontspec
  \defaultfontfeatures{Scale=MatchLowercase}
  \defaultfontfeatures[\rmfamily]{Ligatures=TeX,Scale=1}
\fi
\usepackage{lmodern}
\ifPDFTeX\else
  % xetex/luatex font selection
\fi
% Use upquote if available, for straight quotes in verbatim environments
\IfFileExists{upquote.sty}{\usepackage{upquote}}{}
\IfFileExists{microtype.sty}{% use microtype if available
  \usepackage[]{microtype}
  \UseMicrotypeSet[protrusion]{basicmath} % disable protrusion for tt fonts
}{}
\makeatletter
\@ifundefined{KOMAClassName}{% if non-KOMA class
  \IfFileExists{parskip.sty}{%
    \usepackage{parskip}
  }{% else
    \setlength{\parindent}{0pt}
    \setlength{\parskip}{6pt plus 2pt minus 1pt}}
}{% if KOMA class
  \KOMAoptions{parskip=half}}
\makeatother
\usepackage{xcolor}
\usepackage[margin=1in]{geometry}
\usepackage{color}
\usepackage{fancyvrb}
\newcommand{\VerbBar}{|}
\newcommand{\VERB}{\Verb[commandchars=\\\{\}]}
\DefineVerbatimEnvironment{Highlighting}{Verbatim}{commandchars=\\\{\}}
% Add ',fontsize=\small' for more characters per line
\usepackage{framed}
\definecolor{shadecolor}{RGB}{248,248,248}
\newenvironment{Shaded}{\begin{snugshade}}{\end{snugshade}}
\newcommand{\AlertTok}[1]{\textcolor[rgb]{0.94,0.16,0.16}{#1}}
\newcommand{\AnnotationTok}[1]{\textcolor[rgb]{0.56,0.35,0.01}{\textbf{\textit{#1}}}}
\newcommand{\AttributeTok}[1]{\textcolor[rgb]{0.13,0.29,0.53}{#1}}
\newcommand{\BaseNTok}[1]{\textcolor[rgb]{0.00,0.00,0.81}{#1}}
\newcommand{\BuiltInTok}[1]{#1}
\newcommand{\CharTok}[1]{\textcolor[rgb]{0.31,0.60,0.02}{#1}}
\newcommand{\CommentTok}[1]{\textcolor[rgb]{0.56,0.35,0.01}{\textit{#1}}}
\newcommand{\CommentVarTok}[1]{\textcolor[rgb]{0.56,0.35,0.01}{\textbf{\textit{#1}}}}
\newcommand{\ConstantTok}[1]{\textcolor[rgb]{0.56,0.35,0.01}{#1}}
\newcommand{\ControlFlowTok}[1]{\textcolor[rgb]{0.13,0.29,0.53}{\textbf{#1}}}
\newcommand{\DataTypeTok}[1]{\textcolor[rgb]{0.13,0.29,0.53}{#1}}
\newcommand{\DecValTok}[1]{\textcolor[rgb]{0.00,0.00,0.81}{#1}}
\newcommand{\DocumentationTok}[1]{\textcolor[rgb]{0.56,0.35,0.01}{\textbf{\textit{#1}}}}
\newcommand{\ErrorTok}[1]{\textcolor[rgb]{0.64,0.00,0.00}{\textbf{#1}}}
\newcommand{\ExtensionTok}[1]{#1}
\newcommand{\FloatTok}[1]{\textcolor[rgb]{0.00,0.00,0.81}{#1}}
\newcommand{\FunctionTok}[1]{\textcolor[rgb]{0.13,0.29,0.53}{\textbf{#1}}}
\newcommand{\ImportTok}[1]{#1}
\newcommand{\InformationTok}[1]{\textcolor[rgb]{0.56,0.35,0.01}{\textbf{\textit{#1}}}}
\newcommand{\KeywordTok}[1]{\textcolor[rgb]{0.13,0.29,0.53}{\textbf{#1}}}
\newcommand{\NormalTok}[1]{#1}
\newcommand{\OperatorTok}[1]{\textcolor[rgb]{0.81,0.36,0.00}{\textbf{#1}}}
\newcommand{\OtherTok}[1]{\textcolor[rgb]{0.56,0.35,0.01}{#1}}
\newcommand{\PreprocessorTok}[1]{\textcolor[rgb]{0.56,0.35,0.01}{\textit{#1}}}
\newcommand{\RegionMarkerTok}[1]{#1}
\newcommand{\SpecialCharTok}[1]{\textcolor[rgb]{0.81,0.36,0.00}{\textbf{#1}}}
\newcommand{\SpecialStringTok}[1]{\textcolor[rgb]{0.31,0.60,0.02}{#1}}
\newcommand{\StringTok}[1]{\textcolor[rgb]{0.31,0.60,0.02}{#1}}
\newcommand{\VariableTok}[1]{\textcolor[rgb]{0.00,0.00,0.00}{#1}}
\newcommand{\VerbatimStringTok}[1]{\textcolor[rgb]{0.31,0.60,0.02}{#1}}
\newcommand{\WarningTok}[1]{\textcolor[rgb]{0.56,0.35,0.01}{\textbf{\textit{#1}}}}
\usepackage{graphicx}
\makeatletter
\def\maxwidth{\ifdim\Gin@nat@width>\linewidth\linewidth\else\Gin@nat@width\fi}
\def\maxheight{\ifdim\Gin@nat@height>\textheight\textheight\else\Gin@nat@height\fi}
\makeatother
% Scale images if necessary, so that they will not overflow the page
% margins by default, and it is still possible to overwrite the defaults
% using explicit options in \includegraphics[width, height, ...]{}
\setkeys{Gin}{width=\maxwidth,height=\maxheight,keepaspectratio}
% Set default figure placement to htbp
\makeatletter
\def\fps@figure{htbp}
\makeatother
\setlength{\emergencystretch}{3em} % prevent overfull lines
\providecommand{\tightlist}{%
  \setlength{\itemsep}{0pt}\setlength{\parskip}{0pt}}
\setcounter{secnumdepth}{-\maxdimen} % remove section numbering
\ifLuaTeX
  \usepackage{selnolig}  % disable illegal ligatures
\fi
\IfFileExists{bookmark.sty}{\usepackage{bookmark}}{\usepackage{hyperref}}
\IfFileExists{xurl.sty}{\usepackage{xurl}}{} % add URL line breaks if available
\urlstyle{same}
\hypersetup{
  pdftitle={Are COVID19 new daily deaths per thousand affected by a state's political affiliation?},
  pdfauthor={MG},
  hidelinks,
  pdfcreator={LaTeX via pandoc}}

\title{Are COVID19 new daily deaths per thousand affected by a state's
political affiliation?}
\author{MG}
\date{April 13, 2024}

\begin{document}
\maketitle

\hypertarget{summary}{%
\section{Summary}\label{summary}}

This report analyses the COVID19 Dataset from Center for Systems Science
and Engineering (CSSE) at Johns Hopkins University
\url{https://github.com/CSSEGISandData/COVID-19}.

The dataset contains global COVID19 confirmed cases and deaths, and the
data was collected between the beginning of the pandemic in March of
2020 until March 10, 2023.

This report will focus on USA data only, and will explore if a state's
political affiliation, as determined by presidential vote outcome in
2020 election (i.e.~``red'' - republican vs.~``blue'' - democratic
state), has any effect on COVID19 daily deaths per thousand.

@todo: how to easily viz red vs blue cases/deaths? barplot? @todo: how
to split up data into train vs test sets? - by date? @todo: model =
deaths\_per\_thou \textasciitilde{} blue vs red? @todo: bias sources:
pandemic started in northeast, so in the beginning the numbers there are
much higher new variants, vaccines, generally low number of features
(not using vax vs unvax, mask mandates, etc)

\hypertarget{importing-and-cleaning-the-data}{%
\section{Importing and Cleaning the
data}\label{importing-and-cleaning-the-data}}

\begin{Shaded}
\begin{Highlighting}[]
\FunctionTok{library}\NormalTok{(tidyverse)}
\FunctionTok{library}\NormalTok{(lubridate)}
\FunctionTok{library}\NormalTok{(readxl)}
\FunctionTok{library}\NormalTok{(ggplot2)}
\end{Highlighting}
\end{Shaded}

First, we will load the US data and perform some basic cleanup. We will
use the CSV files for US confirmed cases and deaths.

\begin{Shaded}
\begin{Highlighting}[]
\NormalTok{url\_in }\OtherTok{\textless{}{-}} \StringTok{"https://raw.githubusercontent.com/CSSEGISandData/COVID{-}19/master/csse\_covid\_19\_data/csse\_covid\_19\_time\_series/"}

\NormalTok{file\_names }\OtherTok{\textless{}{-}} \FunctionTok{c}\NormalTok{(}
  \StringTok{\textquotesingle{}time\_series\_covid19\_confirmed\_US.csv\textquotesingle{}}\NormalTok{,}
  \StringTok{\textquotesingle{}time\_series\_covid19\_deaths\_US.csv\textquotesingle{}}
\NormalTok{)}

\NormalTok{urls }\OtherTok{\textless{}{-}} \FunctionTok{str\_c}\NormalTok{(url\_in, file\_names)}

\NormalTok{US\_cases }\OtherTok{\textless{}{-}} \FunctionTok{read\_csv}\NormalTok{(urls[}\DecValTok{1}\NormalTok{])}
\NormalTok{US\_deaths }\OtherTok{\textless{}{-}} \FunctionTok{read\_csv}\NormalTok{(urls[}\DecValTok{2}\NormalTok{])}
\end{Highlighting}
\end{Shaded}

The cases and deaths CSV files contain several columns which we will not
be using, for example: ``UID'', ``Lat'', ``Long'', and others, and also
use a column for each date, so we will remove the unused columns, and
convert the date columns into a single column called ``date'' and
convert their values into a Date object.

\begin{Shaded}
\begin{Highlighting}[]
\NormalTok{US\_cases }\OtherTok{\textless{}{-}}\NormalTok{ US\_cases }\SpecialCharTok{\%\textgreater{}\%} 
  \FunctionTok{pivot\_longer}\NormalTok{(}\AttributeTok{cols =} \SpecialCharTok{{-}}\NormalTok{(UID}\SpecialCharTok{:}\NormalTok{Combined\_Key), }\AttributeTok{names\_to=}\StringTok{"date"}\NormalTok{, }\AttributeTok{values\_to=}\StringTok{"cases"}\NormalTok{) }\SpecialCharTok{\%\textgreater{}\%}
  \FunctionTok{select}\NormalTok{(Admin2}\SpecialCharTok{:}\NormalTok{cases) }\SpecialCharTok{\%\textgreater{}\%}
  \FunctionTok{mutate}\NormalTok{(}\AttributeTok{date =} \FunctionTok{mdy}\NormalTok{(date)) }\SpecialCharTok{\%\textgreater{}\%}
  \FunctionTok{select}\NormalTok{(}\SpecialCharTok{{-}}\FunctionTok{c}\NormalTok{(Lat, Long\_))}

\NormalTok{US\_deaths }\OtherTok{\textless{}{-}}\NormalTok{ US\_deaths }\SpecialCharTok{\%\textgreater{}\%} 
  \FunctionTok{pivot\_longer}\NormalTok{(}\AttributeTok{cols =} \SpecialCharTok{{-}}\NormalTok{(UID}\SpecialCharTok{:}\NormalTok{Population), }\AttributeTok{names\_to=}\StringTok{"date"}\NormalTok{, }\AttributeTok{values\_to=}\StringTok{"deaths"}\NormalTok{) }\SpecialCharTok{\%\textgreater{}\%}
  \FunctionTok{select}\NormalTok{(Admin2}\SpecialCharTok{:}\NormalTok{deaths) }\SpecialCharTok{\%\textgreater{}\%}
  \FunctionTok{mutate}\NormalTok{(}\AttributeTok{date =} \FunctionTok{mdy}\NormalTok{(date)) }\SpecialCharTok{\%\textgreater{}\%}
  \FunctionTok{select}\NormalTok{(}\SpecialCharTok{{-}}\FunctionTok{c}\NormalTok{(Lat, Long\_))}
\end{Highlighting}
\end{Shaded}

Next, we will merge US cases and deaths data into a single dataframe.

\begin{Shaded}
\begin{Highlighting}[]
\NormalTok{US }\OtherTok{\textless{}{-}}\NormalTok{ US\_cases }\SpecialCharTok{\%\textgreater{}\%}
  \FunctionTok{full\_join}\NormalTok{(US\_deaths)}
\end{Highlighting}
\end{Shaded}

Now, let's look at the summary of the combined cases and deaths dataset.

\begin{Shaded}
\begin{Highlighting}[]
\FunctionTok{summary}\NormalTok{(US)}
\end{Highlighting}
\end{Shaded}

\begin{verbatim}
##     Admin2          Province_State     Country_Region     Combined_Key      
##  Length:3819906     Length:3819906     Length:3819906     Length:3819906    
##  Class :character   Class :character   Class :character   Class :character  
##  Mode  :character   Mode  :character   Mode  :character   Mode  :character  
##                                                                             
##                                                                             
##                                                                             
##       date                cases           Population           deaths       
##  Min.   :2020-01-22   Min.   :  -3073   Min.   :       0   Min.   :  -82.0  
##  1st Qu.:2020-11-02   1st Qu.:    330   1st Qu.:    9917   1st Qu.:    4.0  
##  Median :2021-08-15   Median :   2272   Median :   24892   Median :   37.0  
##  Mean   :2021-08-15   Mean   :  14088   Mean   :   99604   Mean   :  186.9  
##  3rd Qu.:2022-05-28   3rd Qu.:   8159   3rd Qu.:   64979   3rd Qu.:  122.0  
##  Max.   :2023-03-09   Max.   :3710586   Max.   :10039107   Max.   :35545.0
\end{verbatim}

From the summary we can see that Population column has rows with 0 as
the value, and there are also rows where cases or deaths are less than
0, so let's remove those rows.

Additionally, since we are using 2020 election to determine political
affiliation of the sates, let's use the data starting January 1, 2021.

\begin{Shaded}
\begin{Highlighting}[]
\NormalTok{US }\OtherTok{\textless{}{-}}\NormalTok{ US }\SpecialCharTok{\%\textgreater{}\%}
  \FunctionTok{filter}\NormalTok{(Population }\SpecialCharTok{\textgreater{}} \DecValTok{0}\NormalTok{, cases }\SpecialCharTok{\textgreater{}=} \DecValTok{0}\NormalTok{, deaths }\SpecialCharTok{\textgreater{}=} \DecValTok{0}\NormalTok{, date }\SpecialCharTok{\textgreater{}=} \StringTok{\textquotesingle{}2021{-}01{-}01\textquotesingle{}}\NormalTok{)}

\FunctionTok{summary}\NormalTok{(US)}
\end{Highlighting}
\end{Shaded}

\begin{verbatim}
##     Admin2          Province_State     Country_Region     Combined_Key      
##  Length:2575146     Length:2575146     Length:2575146     Length:2575146    
##  Class :character   Class :character   Class :character   Class :character  
##  Mode  :character   Mode  :character   Mode  :character   Mode  :character  
##                                                                             
##                                                                             
##                                                                             
##       date                cases           Population           deaths       
##  Min.   :2021-01-01   Min.   :      0   Min.   :      86   Min.   :    0.0  
##  1st Qu.:2021-07-19   1st Qu.:   1722   1st Qu.:   11137   1st Qu.:   26.0  
##  Median :2022-02-03   Median :   4626   Median :   26205   Median :   71.0  
##  Mean   :2022-02-03   Mean   :  20025   Mean   :  103153   Mean   :  253.6  
##  3rd Qu.:2022-08-22   3rd Qu.:  12823   3rd Qu.:   67493   3rd Qu.:  179.0  
##  Max.   :2023-03-09   Max.   :3710586   Max.   :10039107   Max.   :35545.0
\end{verbatim}

Since we will be analyzing data for each of the US states, we will next
create a dataframe with cases and deaths statistics for each state.

\begin{Shaded}
\begin{Highlighting}[]
\NormalTok{US\_by\_state }\OtherTok{\textless{}{-}}\NormalTok{ US }\SpecialCharTok{\%\textgreater{}\%}
  \FunctionTok{group\_by}\NormalTok{(Province\_State, Country\_Region, date) }\SpecialCharTok{\%\textgreater{}\%}
  \FunctionTok{summarize}\NormalTok{(}\AttributeTok{cases =} \FunctionTok{sum}\NormalTok{(cases), }\AttributeTok{deaths =} \FunctionTok{sum}\NormalTok{(deaths), }\AttributeTok{Population =} \FunctionTok{sum}\NormalTok{(Population)) }\SpecialCharTok{\%\textgreater{}\%}
  \FunctionTok{select}\NormalTok{(Province\_State, Country\_Region, date, cases, deaths, Population) }\SpecialCharTok{\%\textgreater{}\%}
  \FunctionTok{ungroup}\NormalTok{()}
\end{Highlighting}
\end{Shaded}

\hypertarget{data-exploration-and-feature-engineering}{%
\section{Data Exploration and Feature
Engineering}\label{data-exploration-and-feature-engineering}}

\hypertarget{adding-deaths-per-thousand-data}{%
\subsection{Adding deaths per thousand
data}\label{adding-deaths-per-thousand-data}}

The numbers for deaths in the dataset are cumulative, so let's see which
states are top 10 by the end of the data collection period.

\begin{Shaded}
\begin{Highlighting}[]
\NormalTok{US\_by\_state }\SpecialCharTok{\%\textgreater{}\%}
  \FunctionTok{group\_by}\NormalTok{(Province\_State) }\SpecialCharTok{\%\textgreater{}\%}
  \FunctionTok{summarize}\NormalTok{(}\AttributeTok{deaths =} \FunctionTok{max}\NormalTok{(deaths), }\AttributeTok{population =} \FunctionTok{max}\NormalTok{(Population)) }\SpecialCharTok{\%\textgreater{}\%}
  \FunctionTok{slice\_max}\NormalTok{(deaths, }\AttributeTok{n =} \DecValTok{10}\NormalTok{)}
\end{Highlighting}
\end{Shaded}

\begin{verbatim}
## # A tibble: 10 x 3
##    Province_State deaths population
##    <chr>           <dbl>      <dbl>
##  1 California     101159   39512223
##  2 Texas           93355   28995881
##  3 Florida         86454   21477737
##  4 New York        76592   19453561
##  5 Pennsylvania    50398   12801989
##  6 Michigan        41964    9986857
##  7 Ohio            41794   11689100
##  8 Georgia         40833   10617423
##  9 Illinois        36431   12671821
## 10 New Jersey      36015    8882190
\end{verbatim}

Sate population varies significantly, and deaths will tend to be higher
in states with larger population, so we will add a new variable
``deaths\_per\_thou'' to have a better way to compare individual state's
numbers.

\begin{Shaded}
\begin{Highlighting}[]
\NormalTok{US\_by\_state }\OtherTok{\textless{}{-}}\NormalTok{ US\_by\_state }\SpecialCharTok{\%\textgreater{}\%}
  \FunctionTok{mutate}\NormalTok{(}\AttributeTok{deaths\_per\_thou =}\NormalTok{ deaths }\SpecialCharTok{*} \DecValTok{1000} \SpecialCharTok{/}\NormalTok{ Population)}
\end{Highlighting}
\end{Shaded}

We can now see what are the top 10 states with highest total deaths per
thousand people.

\begin{Shaded}
\begin{Highlighting}[]
\NormalTok{US\_by\_state }\SpecialCharTok{\%\textgreater{}\%}
  \FunctionTok{group\_by}\NormalTok{(Province\_State) }\SpecialCharTok{\%\textgreater{}\%}
  \FunctionTok{summarize}\NormalTok{(}\AttributeTok{deaths\_per\_thou =} \FunctionTok{max}\NormalTok{(deaths\_per\_thou), }\AttributeTok{population =} \FunctionTok{max}\NormalTok{(Population)) }\SpecialCharTok{\%\textgreater{}\%}
  \FunctionTok{slice\_max}\NormalTok{(deaths\_per\_thou, }\AttributeTok{n =} \DecValTok{10}\NormalTok{)}
\end{Highlighting}
\end{Shaded}

\begin{verbatim}
## # A tibble: 10 x 3
##    Province_State deaths_per_thou population
##    <chr>                    <dbl>      <dbl>
##  1 Arizona                   4.55    7278717
##  2 Mississippi               4.49    2976149
##  3 West Virginia             4.44    1792147
##  4 New Mexico                4.32    2096829
##  5 Arkansas                  4.31    3017804
##  6 Alabama                   4.29    4903185
##  7 Tennessee                 4.21    6829174
##  8 Michigan                  4.20    9986857
##  9 Kentucky                  4.06    4467673
## 10 New Jersey                4.05    8882190
\end{verbatim}

\hypertarget{adding-daily-new-deaths-data}{%
\subsection{Adding daily new deaths
data}\label{adding-daily-new-deaths-data}}

We are specifically interested in daily deaths in each state, so we will
add two new columns ``new\_deaths''.

\begin{Shaded}
\begin{Highlighting}[]
\NormalTok{US\_by\_state }\OtherTok{\textless{}{-}}\NormalTok{ US\_by\_state }\SpecialCharTok{\%\textgreater{}\%}
  \FunctionTok{mutate}\NormalTok{(}\AttributeTok{new\_deaths =}\NormalTok{ deaths }\SpecialCharTok{{-}} \FunctionTok{lag}\NormalTok{(deaths))}
\end{Highlighting}
\end{Shaded}

\begin{Shaded}
\begin{Highlighting}[]
\NormalTok{US\_by\_state }\SpecialCharTok{\%\textgreater{}\%}
  \FunctionTok{filter}\NormalTok{(new\_deaths }\SpecialCharTok{\textgreater{}} \DecValTok{0}\NormalTok{) }\SpecialCharTok{\%\textgreater{}\%}
  \FunctionTok{ggplot}\NormalTok{(}\FunctionTok{aes}\NormalTok{(}\AttributeTok{x =}\NormalTok{ date, }\AttributeTok{y =}\NormalTok{ new\_deaths)) }\SpecialCharTok{+}
  \FunctionTok{geom\_line}\NormalTok{(}\FunctionTok{aes}\NormalTok{(}\AttributeTok{color =} \StringTok{"new\_deaths"}\NormalTok{)) }\SpecialCharTok{+}
  \FunctionTok{geom\_point}\NormalTok{(}\FunctionTok{aes}\NormalTok{(}\AttributeTok{color =} \StringTok{"new\_deaths"}\NormalTok{)) }\SpecialCharTok{+}
  \FunctionTok{scale\_y\_log10}\NormalTok{() }\SpecialCharTok{+}
  \FunctionTok{theme}\NormalTok{(}\AttributeTok{legend.position =} \StringTok{"bottom"}\NormalTok{, }\AttributeTok{axis.text.x =} \FunctionTok{element\_text}\NormalTok{(}\AttributeTok{angle=}\DecValTok{90}\NormalTok{)) }\SpecialCharTok{+}
  \FunctionTok{labs}\NormalTok{(}\AttributeTok{title =} \FunctionTok{str\_c}\NormalTok{(}\StringTok{"COVID19 Daily Deaths (log scale)"}\NormalTok{), }\AttributeTok{y =} \ConstantTok{NULL}\NormalTok{)}
\end{Highlighting}
\end{Shaded}

\includegraphics{covid_19_files/figure-latex/viz_new_deaths-1.pdf}

Now lets add another column, ``new\_deaths\_per\_thou'' to account for
differences in states' population.

\begin{Shaded}
\begin{Highlighting}[]
\NormalTok{US\_by\_state }\OtherTok{\textless{}{-}}\NormalTok{ US\_by\_state }\SpecialCharTok{\%\textgreater{}\%}
  \FunctionTok{mutate}\NormalTok{(}\AttributeTok{new\_deaths\_per\_thou =}\NormalTok{ new\_deaths }\SpecialCharTok{*} \DecValTok{1000} \SpecialCharTok{/}\NormalTok{ Population)}
\end{Highlighting}
\end{Shaded}

\begin{Shaded}
\begin{Highlighting}[]
\NormalTok{US\_by\_state }\SpecialCharTok{\%\textgreater{}\%}
  \FunctionTok{filter}\NormalTok{(new\_deaths\_per\_thou }\SpecialCharTok{\textgreater{}} \DecValTok{0}\NormalTok{) }\SpecialCharTok{\%\textgreater{}\%}
  \FunctionTok{ggplot}\NormalTok{(}\FunctionTok{aes}\NormalTok{(}\AttributeTok{x =}\NormalTok{ date, }\AttributeTok{y =}\NormalTok{ new\_deaths\_per\_thou)) }\SpecialCharTok{+}
  \FunctionTok{geom\_line}\NormalTok{(}\FunctionTok{aes}\NormalTok{(}\AttributeTok{color =} \StringTok{"new\_deaths\_per\_thou"}\NormalTok{)) }\SpecialCharTok{+}
  \FunctionTok{geom\_point}\NormalTok{(}\FunctionTok{aes}\NormalTok{(}\AttributeTok{color =} \StringTok{"new\_deaths\_per\_thou"}\NormalTok{)) }\SpecialCharTok{+}
  \FunctionTok{scale\_y\_log10}\NormalTok{() }\SpecialCharTok{+}
  \FunctionTok{theme}\NormalTok{(}\AttributeTok{legend.position =} \StringTok{"bottom"}\NormalTok{, }\AttributeTok{axis.text.x =} \FunctionTok{element\_text}\NormalTok{(}\AttributeTok{angle=}\DecValTok{90}\NormalTok{)) }\SpecialCharTok{+}
  \FunctionTok{labs}\NormalTok{(}\AttributeTok{title =} \FunctionTok{str\_c}\NormalTok{(}\StringTok{"COVID19 Daily Deaths Per Thousand (log scale)"}\NormalTok{), }\AttributeTok{y =} \ConstantTok{NULL}\NormalTok{)}
\end{Highlighting}
\end{Shaded}

\includegraphics{covid_19_files/figure-latex/viz_new_deaths_per_thou-1.pdf}

\hypertarget{adding-state-party-affiliation}{%
\subsection{Adding state party
affiliation}\label{adding-state-party-affiliation}}

Next, we will add a new field to mark a state as ``red'' or ``blue''. We
will official data from the Federal Election Commission. The full
dataset is an Excel spreadsheet, available here:
\url{https://www.fec.gov/documents/4228/federalelections2020.xlsx}. The
spreadsheet includes multiple sheets, we will be using sheet 9, ``2020
Pres General Results''. We will also remove spaces from the column
names.

\begin{Shaded}
\begin{Highlighting}[]
\NormalTok{election\_results }\OtherTok{\textless{}{-}} \FunctionTok{read\_excel}\NormalTok{(}\StringTok{\textquotesingle{}./federalelections2020.xlsx\textquotesingle{}}\NormalTok{, }\AttributeTok{sheet =} \DecValTok{9}\NormalTok{)}

\CommentTok{\# remove spaces from column names}
\FunctionTok{names}\NormalTok{(election\_results) }\OtherTok{\textless{}{-}} \FunctionTok{make.names}\NormalTok{(}\FunctionTok{names}\NormalTok{(election\_results), }\AttributeTok{unique =} \ConstantTok{TRUE}\NormalTok{)}

\FunctionTok{colnames}\NormalTok{(election\_results)}
\end{Highlighting}
\end{Shaded}

\begin{verbatim}
##  [1] "X1"                            "FEC.ID"                       
##  [3] "STATE"                         "STATE.ABBREVIATION"           
##  [5] "GENERAL.ELECTION.DATE"         "FIRST.NAME"                   
##  [7] "LAST.NAME"                     "LAST.NAME...FIRST"            
##  [9] "TOTAL.VOTES"                   "PARTY"                        
## [11] "GENERAL.RESULTS"               "GENERAL.."                    
## [13] "TOTAL.VOTES.."                 "COMBINED.GE.PARTY.TOTALS..NY."
## [15] "COMBINED....NY."               "WINNER.INDICATOR"             
## [17] "ELECTORAL.VOTES"               "FOOTNOTES"
\end{verbatim}

We will be using the ``WINNER.INDICATOR'' column with value of ``W'' or
``W*'' (for Maine) to get the winner for each state, and the ``PARTY''
column to determine the party affiliation for that winner. ``STATE''
column will be used to map the winner and their party to a specific
state.

\begin{Shaded}
\begin{Highlighting}[]
\NormalTok{election\_results\_clean }\OtherTok{\textless{}{-}}\NormalTok{ election\_results }\SpecialCharTok{\%\textgreater{}\%}
  \FunctionTok{select}\NormalTok{(STATE, PARTY, WINNER.INDICATOR) }\SpecialCharTok{\%\textgreater{}\%}
  \FunctionTok{filter}\NormalTok{(WINNER.INDICATOR }\SpecialCharTok{\%in\%} \FunctionTok{c}\NormalTok{(}\StringTok{\textquotesingle{}W\textquotesingle{}}\NormalTok{, }\StringTok{\textquotesingle{}W*\textquotesingle{}}\NormalTok{))}
\end{Highlighting}
\end{Shaded}

``PARTY'' column has several different values, not just ``D'' or ``R'':

\begin{Shaded}
\begin{Highlighting}[]
\FunctionTok{unique}\NormalTok{(election\_results\_clean}\SpecialCharTok{$}\NormalTok{PARTY)}
\end{Highlighting}
\end{Shaded}

\begin{verbatim}
## [1] "R"                 "D"                 "DFL"              
## [4] "Combined Parties:" "WF"
\end{verbatim}

``DFL'' is the Minnesota Democratic Party
(\url{https://dfl.org/about/}), so we can treat value ``DFL'' as ``D''.
New York state's ``WF'' is the Working Families party
(\url{https://workingfamilies.org/state/new-york/}) and is also a
Democratic party, so we can omit that row, since New York already has a
winner entry with value ``D''. Additionally, ``Combined Parties:'' for
New York appears to be a special indicator to mark both Democratic and
Working Families parties as the winner, so we can also omit this row.
Finally, we will drop ``WINNER.INDICATOR'' column, and will rename
values of ``D'' to ``blue'', and ``R'' to ``red'' to make the results
more readable for the final analysis.

\begin{Shaded}
\begin{Highlighting}[]
\NormalTok{election\_results\_clean }\OtherTok{\textless{}{-}}\NormalTok{ election\_results\_clean }\SpecialCharTok{\%\textgreater{}\%}
  \FunctionTok{filter}\NormalTok{(PARTY }\SpecialCharTok{\%in\%} \FunctionTok{c}\NormalTok{(}\StringTok{\textquotesingle{}D\textquotesingle{}}\NormalTok{, }\StringTok{\textquotesingle{}R\textquotesingle{}}\NormalTok{, }\StringTok{\textquotesingle{}DFL\textquotesingle{}}\NormalTok{)) }\SpecialCharTok{\%\textgreater{}\%}
  \FunctionTok{mutate}\NormalTok{(}\AttributeTok{PARTY =} \FunctionTok{recode}\NormalTok{(PARTY, }\AttributeTok{DFL =} \StringTok{\textquotesingle{}blue\textquotesingle{}}\NormalTok{, }\AttributeTok{R =} \StringTok{\textquotesingle{}red\textquotesingle{}}\NormalTok{, }\AttributeTok{D =} \StringTok{\textquotesingle{}blue\textquotesingle{}}\NormalTok{)) }\SpecialCharTok{\%\textgreater{}\%}
  \FunctionTok{select}\NormalTok{(}\SpecialCharTok{{-}}\NormalTok{WINNER.INDICATOR)}
\end{Highlighting}
\end{Shaded}

Next, we will merge the political affiliation data with our COVID19 by
state dataset.

\begin{Shaded}
\begin{Highlighting}[]
\NormalTok{US\_by\_state }\OtherTok{\textless{}{-}}\NormalTok{ US\_by\_state }\SpecialCharTok{\%\textgreater{}\%}
  \FunctionTok{full\_join}\NormalTok{(election\_results\_clean, }\AttributeTok{by =} \FunctionTok{join\_by}\NormalTok{(Province\_State }\SpecialCharTok{==}\NormalTok{ STATE))}
\end{Highlighting}
\end{Shaded}

Now we can visualize the changes in daily deaths per thousand broken out
by Democratic vs.~Republican states

\begin{Shaded}
\begin{Highlighting}[]
\NormalTok{US\_by\_state }\SpecialCharTok{\%\textgreater{}\%}
  \FunctionTok{filter}\NormalTok{(new\_deaths\_per\_thou }\SpecialCharTok{\textgreater{}} \DecValTok{0}\NormalTok{, PARTY }\SpecialCharTok{\%in\%} \FunctionTok{c}\NormalTok{(}\StringTok{\textquotesingle{}red\textquotesingle{}}\NormalTok{, }\StringTok{\textquotesingle{}blue\textquotesingle{}}\NormalTok{)) }\SpecialCharTok{\%\textgreater{}\%}
  \FunctionTok{ggplot}\NormalTok{(}\FunctionTok{aes}\NormalTok{(}\AttributeTok{x =}\NormalTok{ date, }\AttributeTok{y =}\NormalTok{ new\_deaths\_per\_thou, }\AttributeTok{color =}\NormalTok{ PARTY)) }\SpecialCharTok{+}
  \CommentTok{\#geom\_point() +}
  \FunctionTok{geom\_smooth}\NormalTok{(}\AttributeTok{method =} \StringTok{"lm"}\NormalTok{, }\AttributeTok{se =} \ConstantTok{FALSE}\NormalTok{, }\FunctionTok{aes}\NormalTok{(}\AttributeTok{group =}\NormalTok{ PARTY)) }\SpecialCharTok{+}
  \FunctionTok{scale\_color\_manual}\NormalTok{(}\AttributeTok{values =} \FunctionTok{c}\NormalTok{(}\StringTok{"red"} \OtherTok{=} \StringTok{"red"}\NormalTok{, }\StringTok{"blue"} \OtherTok{=} \StringTok{"blue"}\NormalTok{)) }\SpecialCharTok{+}
  \FunctionTok{scale\_y\_log10}\NormalTok{() }\SpecialCharTok{+}
  \FunctionTok{theme}\NormalTok{(}\AttributeTok{legend.position =} \StringTok{"bottom"}\NormalTok{, }\AttributeTok{axis.text.x =} \FunctionTok{element\_text}\NormalTok{(}\AttributeTok{angle =} \DecValTok{90}\NormalTok{)) }\SpecialCharTok{+}
  \FunctionTok{labs}\NormalTok{(}\AttributeTok{title =} \StringTok{"Daily deaths per thousand by Republican vs. Democratic states"}\NormalTok{, }\AttributeTok{y =} \ConstantTok{NULL}\NormalTok{)}
\end{Highlighting}
\end{Shaded}

\includegraphics{covid_19_files/figure-latex/viz_new_deaths_blue_vs_red-1.pdf}

\end{document}
